%%%%%%%%%%%%%%%%%%%%%%%%%%%%%%%%%%%%%%%%%
% Journal Article
% LaTeX Template
% Version 1.3 (9/9/13)
%
% This template has been downloaded from:
% http://www.LaTeXTemplates.com
%
% Original author:
% Frits Wenneker (http://www.howtotex.com)
%
% License:
% CC BY-NC-SA 3.0 (http://creativecommons.org/licenses/by-nc-sa/3.0/)
%
%%%%%%%%%%%%%%%%%%%%%%%%%%%%%%%%%%%%%%%%%

%----------------------------------------------------------------------------------------
%	PACKAGES AND OTHER DOCUMENT CONFIGURATIONS
%----------------------------------------------------------------------------------------

\documentclass[twoside]{article}

\usepackage{lipsum} % Package to generate dummy text throughout this template
%\usepackage[showframe]{geometry}% http://ctan.org/pkg/geometry

\usepackage[sc]{mathpazo} % Use the Palatino font
\usepackage[T1]{fontenc} % Use 8-bit encoding that has 256 glyphs
\linespread{1.05} % Line spacing - Palatino needs more space between lines
\usepackage{microtype} % Slightly tweak font spacing for aesthetics
\usepackage{graphicx}
\usepackage[hmarginratio=1:1,top=32mm,columnsep=20pt]{geometry} % Document margins
\usepackage{multicol} % Used for the two-column layout of the document
\usepackage[hang, small,labelfont=bf,up,textfont=it,up]{caption} % Custom captions under/above floats in tables or figures
\usepackage{booktabs} % Horizontal rules in tables
\usepackage{float} % Required for tables and figures in the multi-column environment - they need to be placed in specific locations with the [H] (e.g. \begin{table}[H])
\usepackage{hyperref} % For hyperlinks in the PDF
\usepackage{amsmath}
\usepackage{lettrine} % The lettrine is the first enlarged letter at the beginning of the text
\usepackage{paralist} % Used for the compactitem environment which makes bullet points with less space between them
\usepackage{authblk}

\usepackage{abstract} % Allows abstract customization
\renewcommand{\abstractnamefont}{\normalfont\bfseries} % Set the "Abstract" text to bold
\renewcommand{\abstracttextfont}{\normalfont\small\itshape} % Set the abstract itself to small italic text
%\usepackage{apacite}
\bibliographystyle{ieeetr}

\usepackage{titlesec} % Allows customization of titles
\renewcommand\thesection{\Roman{section}} % Roman numerals for the sections
\renewcommand\thesubsection{\Roman{subsection}} % Roman numerals for subsections
\titleformat{\section}[block]{\large\scshape\centering}{\thesection.}{1em}{} % Change the look of the section titles
\titleformat{\subsection}[block]{\large}{\thesubsection.}{1em}{} % Change the look of the section titles

\usepackage{fancyhdr} % Headers and footers
\pagestyle{fancy} % All pages have headers and footers
\fancyhead{} % Blank out the default header
\fancyfoot{} % Blank out the default footer
\fancyhead[C]{}
%Running title $\bullet$ November 2012 $\bullet$ Vol. XXI, No. 1} % Custom header text
\fancyfoot[RO,LE]{\thepage} % Custom footer text

%----------------------------------------------------------------------------------------
%	TITLE SECTION
%----------------------------------------------------------------------------------------

\title{\vspace{-15mm}\fontsize{24pt}{10pt}\selectfont\textbf{BioJS-HGV Viewer: Genetic Variation Visualizer}} % Article title

%\author{
%\large
%\textsc{Saket K Choudhary}\thanks{Molecular and Computational Biology, University of Southern California, Los Angeles, California, USA}\\[2mm] % Your name
%\textsc{Leyla J Garcia}\thanks{European Bioinformatics Institute EMBL-EBI, Hinxton, Cambridge, CB10 1SD, UK} \textsc{Andrew Nightingale}\thanks{European Bioinformatics Institute EMBL-EBI, Hinxton, Cambridge, CB10 1SD, UK}\\[2mm] % Your name
%%\normalsize University of California \\ % Your institution
%\normalsize \href{mailto:skchoudh@usc.edu}{skchoudh@usc.edu} % Your email address
%\vspace{-5mm}
%}

\author[1]{Saket Choudhary\thanks{skchoudh@usc.edu}} %\thanks{Molecular and Computational Biology, University of Southern California}}
\author[2]{Leyla Garcia} %\thanks{European Bioinformatics Institute EMBL-EBI, Hinxton, Cambridge, CB10 1SD, UK}}
\author[2]{Andrew Nightingale} %\thanks{European Bioinformatics Institute EMBL-EBI, Hinxton, Cambridge, CB10 1SD, UK}}

\affil[1]{Molecular and Computational Biology, University of Southern California, Los Angeles, USA} %Department of Computer Science, \LaTeX\ University}
\affil[2]{European Bioinformatics Institute EMBL-EBI, Cambridge, England} %Department of Mechanical Engineering, \LaTeX\ University}
\date{}

%----------------------------------------------------------------------------------------

\begin{document}

\maketitle % Insert title

\thispagestyle{fancy} % All pages have headers and footers

%----------------------------------------------------------------------------------------
%	ABSTRACT
%----------------------------------------------------------------------------------------

\begin{abstract}

\noindent % \lipsum[1] % Dummy abstract text

%Genomic studies have resulted in  catalogs of genetic variants in humans. 
Studying the pattern of genetic variants can help us understand a individuals within a population, identify key \textit{'driver variants'} that affect disease states and evolution. Catalogs of genetic variants are vast but lack an interactive exploratory interface.

Here we present BioJS-HGV Viewer, a BioJavascript component to represent and visualize genetic variants pooled from various sources. The tool displays sequences and variants at different levels facilitating representation of variant sites and annotations in a user friendly and interactive manner. 

The code for BioJS-HGV Viewer is available at:\\ \url{https://github.com/saketkc/biojs-genetic-variation-viewer}.

A demo is available at: \url{http://saketkc.github.io/biojs}

\end{abstract}

%----------------------------------------------------------------------------------------
%	ARTICLE CONTENTS
%----------------------------------------------------------------------------------------

\begin{multicols}{2} % Two-column layout throughout the main article text

\section{Introduction}

\lettrine[nindent=0em,lines=3]{W} 
%orem ipsum dolor sit amet, consectetur adipiscing elit.
%\lipsum[2-3] % Dummy text
With the advent of next-generation sequencing technologies, it has been possible to profile genomes in larger numbers. One of the chief outcomes of such projects has been the cataloguing of genetic variants in database resources such as dbSNP\cite{Smigielski2000} and Catalogue Of Somatic Mutations In Cancer (COSMIC)\cite{Forbes2011}. These catalogs contain publicly accessible sets of genetic variants found in species such as mouse, rat, zebrafish and human which can be utilized to study evolutionary relationships, population diversity and disease specific variations. The COSMIC database is a curated set of somatic mutations as observed in cancer samples. Where the current release (v70) contains 1,564,699 unique variants taken from 1,029,547 samples covering 28,735 genes.  dbSNP build 142 contains two orders of magnitude more human variants than COSMIC with 112 million refSNPs\cite{ncbiweb}. The availability of data at such a large scale makes the analysis and interpretation challenging.

Any exploratory attempt at analyzing the variation data would involve visualizing the variants across the genome to determine specific sites, if any, where the mutations are more frequent or are absent completely. Thus, visualization is critical from the point of interpretation of the vast catalogs of variants. \textit{You don't want to piss off COSMIC they will most probably use you tool} There are variant visualization available; for example the COSMIC Genome Browser\cite{Forbes2011} offers a view with limited flexibility to customize the view and annotations for each variant.


 BioJS-HGV Viewer is a BioJS \cite{Corpas2014} component developed to visualize genetic variants in a comprehensive manner. BioJS is an open source javascript library providing various components to visualize biological data. The visualizations are web based and hence are absolutely platform independent.
 
 
%------------------------------------------------

\section{Methods}
The functionality provided by BioJS-HGV Viewer has two mode views:
\begin{compactitem}
\item Overview
\item Detailed or Zoomed View
\end{compactitem}
The architecture of this component is designed to handle both DNA and protein variants. The current implementation makes use of protein variants. These variant sites have been generated by an un-published webservice made available through EBI. This service has an indexed database of protein variants as reported in the COSMIC and UniProtKB\cite{Wu2006} database and is made available as a JSON\cite{json} file. With support for standard data formats, such as VCF,\cite{vcf} being implemented.



The demo at \url{http://saketkc.github.io/biojs} loads the variants for protein \textit{J3KP33}, by default. The component however allows loading other proteins by passing an additional argument to the url. For example: \url{http://saketkc.github.io/biojs/src/test/javascript/TestHGVViewer.html?q=P00533}.

By default, SIFT and Polyphen scores are averaged and the type of mutations are then decided based on this average score. The component however allows user to choose from either or all of the scores.

The user can also choose to hide a particular category of mutations. Both the overview and detailed mode have another \textit{'open view'}\ref{openview} where these mutations can further be separately visualized as \textit{Stop Gained}, \textit{Missense} and \textit{Splice Region}.

All the visualizations are rendered as scalable vector graphics(SVG) using the \textit{d3js}\cite{d3js} javascript library


\subsection{Overview Mode}
In the default mode the viewer presents variant information in a condensed format using a stacked bar char displaying the number and \textit{type of mutations} at each site. The detailed annotations are displayed on hovering over the rectangle as a \textit{tooltip}.
The \textit{type of mutations} are classified as:
\begin{compactitem}
\item \textbf{Benign} %: Mutations that do not metastasize  and are virtually harmless 
\item \textbf{Damaging} %: Mutations that lead to 
\item \textbf{Mixed}
\end{compactitem}

The 'Mixed' category represents an \textbf{intermediate} state between damaging and benign.


The classification currently uses the predictions scores of Polyphen\cite{Ramensky2002} and SIFT\cite{Kumar2009}. Polyphen scores are on a scale of $[0,1]$ with $1$ indicating that the mutation is damaging and $0$ indicating the mutation being benign. SIFT scores also operate on the scale of $[0,1]$ however $0$ indicates a damaging mutation.
The webservice has a database of all mutations across various proteins with pre-generated scores which can be retrieved as a JSON file.

The data thus received is parsed for calculating the number of mutations in each category. Each category is defined by threshold levels. For example a Polyphen score between 0.75 and 1.0 can be considered to reflect a damaging mutation. These threshold levels can be modified by the user. The height of each rectangular box depicting the mutation is dynamically adjusted based on the maximum number of variants at any site.

%By default, the SIFT and Polyphen scores are averaged to generate a combined prediction score. 

\subsection{Detailed View}
In the detailed view\ref{zoomedtooltip} each individual amino acid on the protein is displayed as a rectangular box with all variants at that site, the height of the rectangle being proportional to the reported frequency. The box for variants is colored based on its type. On a \textit{mouse over} action at the variant box, the tooltip shows detailed information about that particular mutation.

%------------------------------------------------

\section{Discussion}
There has been a lack of interactive tools to visualize catalog of mutations comprehensively. BioJS-HGV Viewer is an open source BioJS component that can be used as tool to visualize variants in a flexible manner. Thus, BioJS-HGV Viewer can be a powerful tool for visual interpretation of mutation data. Being entirely web based, it is platform independent.


%----------------------------------------------------------------------------------------
%	REFERENCE LIST
%----------------------------------------------------------------------------------------


\section{Acknowledgments}
We would like to thank the BioJS community for insightful discussions. This project was funded by Google Summer of Code 2014.

\bibliography{biojs_paper}


%----------------------------------------------------------------------------------------



\end{multicols}

\end{document}
